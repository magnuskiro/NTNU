% latex article template

% cheat sheet(eng): http://www.pvv.ntnu.no/~walle/latex/dokumentasjon/latexsheet.pdf
% cheat sheet2(eng): http://www.pvv.ntnu.no/~walle/latex/dokumentasjon/LaTeX-cheat-sheet.pdf
% reference manual(eng): http://ctan.uib.no/info/latex2e-help-texinfo/latex2e.html

% The document class defines the type of document. Presentation, article, letter, etc. 
\documentclass[12pt, a4paper]{article}

% packages to be used. needed to use images and such things. 
\usepackage[pdfborder=0 0 0]{hyperref}
\usepackage[utf8]{inputenc}
\usepackage[english]{babel}
\usepackage{graphicx}
\PassOptionsToPackage{hyphens}{url}

% hides the section numbering. 
\setcounter{secnumdepth}{-1}

% Graphics/image lications and extensions. 
\DeclareGraphicsExtensions{.pdf, .png, .jpg, .jpeg}
\graphicspath{{./images/}}

% Title or header for the document. 
\title{
	Medarbeidersamtale, Øving 2, Ledelse i Praksis
}
% Author
\author{
	Magnus L Kirø \\
	IT-Sjef, Studentmediene i Trondheim 
}
\date{\today}

\begin{document}
\maketitle
\pagenumbering{arabic}

\section{Medarbeidersamtalen}
Medarbeidersamtalen er en samtale for medarbeideren hvor man
kartlegger personens utvikling, situasjon og forventninger. Det er der man
setter mål for medarbeideren til å holde, og finner ut om personen har
tilbakemeldinger eller problemer som må løses. 
Man bruker også medarbeidersamtale til å formidle visjon og strategi.

\section{Gjennomføring}
Man trenger ikke en spesifikk grunn for å holde medarbeidersamtale. Det er noe
man burde holde for å forbedre persones arbeidsopplevelse og produktivitet.
Man burde ha medarbeidersamtale 2 til 4 ganger i året. 

Det er viktig å ta for seg temaer som personlig forhold, familieforhold,
arbeidsforhold, måloppnåelse, psykisk og fysisk helse.

Det fins situasjoner der det kan være gunstig å være flere på
medarbeidersamtalen, men som oftest er det best å bare være to. I tilfeller
hvor det er gunstig med flere parter er der man trenger spesiell kompetanse
eller hvor personen er spesielt vanskelige stilt. 

Hvordan? (forberedelse, etterarbeid, oppfølging)
Man bør forberede seg god før medarbeidersamtale og ha god kontroll på hva og
hvordan man skal gjøre ting. På forhånd bør man ha kartlagt ting som
måloppnåelse, presteringer og grunnleggende helse. Både person og leder må
forberede seg til medarbeidersamtalen og tenke igjennom temaene før samtalen.
Dette fører til en bedre og mer givende samtale hvor man oppnår formålet med
samtalen. 

Man må også ha et klart formål for samtalen klart på forhånd.

Etter samtalen bør man renskrive notatene sine og skrive ned ettertanker fra
samtalen. Dette føres så inn i personaljournalen til personen. Dette er gunstig
i sammenheng med oppfølgig, som skjer ved neste medarbeidersamtale eller
tidliger ved behov.   

Hvilke resultater kan man
forvente av en medarbeidersamtale?
Man bør ikke forvente noen resultater fra samtalen, men man kan forvente å lære
noe nytt. Det beste resultatet man kan oppnå er at medarbeidere føler seg
ivaretatt, hørt og motivert. Om lederen får nyttig innformasjon ut av samtalen
er dette er pluss. 

\section{Personlige erfaringer}
Personlig har jeg dårlig erfaring med medarbeidersamtaler. Jeg har fått lite ut
av det i begge roller. Men jeg kan tenke meg at jeg kan får mer ut av samtalene
i fremtiden. 

Medarbeidersamtalene i min gruppe relaterer seg mye om arbeidet som blir gjort
og den sosiale deltagelsen i gruppa. IT folk er som oftest veldig autonome og i
den gruppen som ikke skjønner hva man får ut av en medarbeidersamtale. Dette
gjør at det ikke er helt lett å noe nyttig ut av samtalene. 

Resultater og personlig utvikling er spessielt viktig. Det er det vi måler mye
av. Man ser som oftest om det er noen som ikke er fornøyd med situasjonen i
gruppa eller sin sosiale tilhørighet.  

Samtelene skal være uformelle. Men de har en tendens til å bli litt for rigide
og ikke så formelle som de burde være. Dette har en del med folks oppfattelse
av medarbeidersamtaler å gjøre. De uformelle samtalene er ofte de som blir
best. Dette er fordi man senker terskelen for åpenhet dermed viser større
tillit, noe som forbedrer medarbeidersamtalen.  

Ved gjennomføring av medarbeidersamtaler forbereder jeg meg tilstrekkelig, ved
å finne tilbakemeldinger til personen og finne positive og negative eksempler å
trekke fram. Vidre så prøver jeg å finne ut hvordan personen er, slik at jeg
kan sørge for at samtalen blir så god tilpasset som mulig.  
Jeg sender også ut liste over det vi skal gå igjennom, slik at personen kan
forberede seg.

Under samtalen tar jeg notater, og skriver ned det viktigste. Jeg prøver å
holde en god tone. Og prøver å holde fokus på samtalen mens jeg lar personen
prate mest mulig.  

Det er også viktig å holde fokus på agendaen og være strukturert i samtalen,
slik at man ikke glemmer å snakke om noe. 

Etter samtalen leser jeg over notatene og skriver ned det viktigste i
personalmappen. 

\section{Vanskelig Team}
Det finnes helt klart temaer som er vanskelige å ta opp. Dette er ting som
personlig forhold og privatliv. Det kan også være negative ting slik som
uholdbare resultater. 

Det er ikke vanskelige å havne i situasjoner hvor man kommer over et vanskelige
tema. Det er nesten uungåelig. Oftest er det situasjoner hvor personen ikke har
gjort jobben sin godt nok.   

Som forberedelser til slike situasjoner vil jeg ta ekstra forhåndsregler og
vite hva jeg sankker om. Har man faktaene på plass, så er det bare den ander
personens innspill som har noe å si. Da blir det lettere å holde godt fokus og
man slipper unødvendige unnskyldninger.

Jeg vil også prøve å finne grunnen til den observeret adferden, slik at dette
kan løses på best mulig måte. 

\section{Teambygging}
Man kan bruke medarbeidersamtalen til team bygging ved at man fokuserer på
teamets behov og formål. 
Man bør også ha fokus på personens oppfatning og verdier, slik at man kan
vidreformidle verdiene og ønskene til hver enkelt inn i teamet. 

Det hjelper også at man får detaljkunnskap om hvert individ i gruppa for å
kunne tilpasse gruppa best mulig og dermed oppnå best resultater. 

Vidre er gode forberedelser et nøkkelelement til å formidle visjon og strategi
til den enkelte. 

Det er også en fordel å ha et eget punkt på teambygging under samtalen, slik
at man kan luke ut problemer og finne nye løsninger på problemer og konflikter
i teamet.  

\end{document}
This is never printed

