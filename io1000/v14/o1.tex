% latex article template

% cheat sheet(eng): http://www.pvv.ntnu.no/~walle/latex/dokumentasjon/latexsheet.pdf
% cheat sheet2(eng): http://www.pvv.ntnu.no/~walle/latex/dokumentasjon/LaTeX-cheat-sheet.pdf
% reference manual(eng): http://ctan.uib.no/info/latex2e-help-texinfo/latex2e.html

% The document class defines the type of document. Presentation, article, letter, etc. 
\documentclass[12pt, a4paper]{article}

% packages to be used. needed to use images and such things. 
\usepackage[pdfborder=0 0 0]{hyperref}
\usepackage[utf8]{inputenc}
\usepackage[english]{babel}
\usepackage{graphicx}
\PassOptionsToPackage{hyphens}{url}

% hides the section numbering. 
\setcounter{secnumdepth}{-1}

% Graphics/image lications and extensions. 
\DeclareGraphicsExtensions{.pdf, .png, .jpg, .jpeg}
\graphicspath{{./images/}}

% Title or header for the document. 
\title{
	Målsetninger, Øving 1, Ledelse i Praksis
}
% Author
\author{
	Magnus L Kirø \\
	IT-Sjef, Studentmediene i Trondheim 
}
\date{\today}

\begin{document}
\maketitle
\pagenumbering{arabic}

\section{Måloppnåelse}
%* Har du nådd noen av dine personlige målsetninger? 
Mine perosnlige målsetninger har kommet langt. Grunnlaget for den neste
IT-sjefen begynner å bli bra. Samtidig som organisasjonen nesten har kommet i
mål med overgangen til ny innfrastruktur.

De utfordringen og problemene vi sitter igjen med er de eldste systemene. Hvor
vi hovedsaklig venter på at nye systemer skal komme på plass og erstatte de
gamle.  

Målsetningen om å få organisasjonen til fungere bedre og mer organisert i
ledelsen fungerte bra helt til vi fikk ny ledelse over jul. Da begynte arbeidet
omigjen. Nye folk, det samme må læres bort på nytt. 

%* Er disse målsetningene i dag de samme som for et halvt år siden, eller har de endret seg? 
Noen av målsetningen vedvarer enda, mens andre har fått større fokus.
Overgangen til ny innfrastruktur er ferdig som målsetning. Dette har gått over
til å bli 'utfasing av gamle systemer'. Noe som ikke blir ferdig mens jeg er
leder. 

Omstruktureringen av organisasjonener er over, men vi sliter fortsatt med
etterdønningene. Dette gjespeiler seg mest i strukturen på ledelsen og
sammensetningen av ledelsen og arbeidsfordelingen innad. Det er ikke alle
oppgavene som er klart definer og heller ikke ansvarsfordelingen. 

%* Hvor vil du legge fokus framover?
Fokuset framover vil ligge på erfaringsoverføring og tilrettelegging for
framtiden. God dokumentasjon og gode retningslinjer for arbeid og beslutninger
er viktig for framtidig stabilitet og drift av organisasjonen. 

Personlig vil jeg ha fokus på å rydde opp og avslutte arbeidet mitt.
Fullføre uferdige oppgaver og overføre det jeg ser at jeg ikke rekker å
fullføre.

\section{Måloppnåelse}
%* Hvordan har det gått med gjengen/seksjonen sine målsetninger? 
Avdelingens mål har blitt nådd, til dels. Avdelingen har blitt en sosial
gruppering som jobber godt sammen. De jobber også godt opp mot resten av
organisasjonen der det er mulig. 

%* Har det dukket opp nye utfordringer i løpet av forrige semester?
Utfordringen er mye de samme nå, som før, men det har kommet nye utfordringer
inn i bildet også. Noe man ikke trodde var en utfordring tidligere har vist seg
å bli en utfordring etterhvert. Spesielt rundt it-drift. 

De vanlige utfordringene ved at andre deler av organisasjonen setter
urealistiske krav er fortsatt tilstedeværende og det vil de antagelig alltid
være også.  

\section{Faglig Utbytte}
%* Hva har du fått ut av faget Ledelse i praksis så langt?
Av faget har jeg fått gode innspill på hvordan man ter seg som en leder og
hvordan man kan ta problemer på en saklig måte. Jeg har fått reflektert over
min lederstil og hva hvilke egenskaper jeg kan forbedre. Lederstilen min er
også noe jeg har fått innspil på, og noe jeg kommer til å utvikle etterhvert. 

\section{Framgang}
%* Ut i fra den tid som har gått siden første øving i faget, har du kommet noe videre
%med problemstilling til avsluttende oppgave? 
Fra første forelesning har jeg kommet langt. Jeg har lært en del om meg selv,
men mest av alt, så har organisasjonen kommet langt. Det er fortsatt mye å ta
tak i, men det er mye som har kommet på plass også. Antagelig en god del som
faller på plass i løpet av semesteret også. 

Når det gjelser problemstilling til oppgaven tenker jeg at det må bli noe sånt
som "IT i endring. Omorganisering, integrering og kommunikasjon.".

%* Hvilke tema i Teamet har vert spesielt viktig for deg dette semesteret og hvorfor? 
Det er vanskelig å si hvilket tema som har vært viktigst, men det so mer lett å
si noe om er hvilke utfordringer som har vært viktigst for meg.

Hovedsaklig er det utfordringen med å samle avdelingen, og få den til å
fungere. Her har organisasjonen som helhet og avdelingen måttet komme til
kompromi på flere områder. Avdelingen har kommet greit ut av det med å ignorere
elementer fra overordnet organisering som har vært til hinder. Dette gjelder
som oftest felles møter hvor det ikke kommer noe nyttig informasjon. All nyttig
informasjon kommer uansett på avdelingsmøte. 

Deretter må det være samhandlingen med resten av organisasjonen. Noe som
fortsatt er en utfordring. Det er mye mail som kommer på tvers av avdelingene,
men det er lite som tilsier at vi kommer noe nærmere hverandre. Dette er et
todelt problem, hvor IT på sin side er mindre flinke til å informere om det som
skjer og vhilke problemer man har, mens man på redaksjonell side er for vandt
til å at ting bare fungerer og man tar derfor ikke så mye kontakt som man
burde.  

Ulempen med en faglig splittet organisasjon er at
visse fagmiljøer ikke integreres med resten på en god måte. Her er det et
klassisk eksempel på hvordan IT og redaksjonelt ikke har overlappende
arbeidsoppgaver eller felles arbeidssted. 

Temaer for utfordring er kommunikkasjon, sammarbeid og tildels roller.
Roller mest i ledelsen, hvor en del må bruke litt tid på å lære seg rollen sin
og finne ut hvordan den fungerer godt over tid. 
Kommunikasjon og sammarbeid går mye på det som er beskrevet over, men gjelder
ikke bare innad i organisasjonen. Det gjelder også utad, mot samfundet og
andre. IT-avdelingen vil å"sikt sammarbeide mer med de andre gjengene på
Samfundet og de andre IT-miljøene på NTNU. Her har vi massivt
forbedringspotensiale.  

\end{document}
This is never printed
