% latex article template

% cheat sheet(eng): http://www.pvv.ntnu.no/~walle/latex/dokumentasjon/latexsheet.pdf
% cheat sheet2(eng): http://www.pvv.ntnu.no/~walle/latex/dokumentasjon/LaTeX-cheat-sheet.pdf
% reference manual(eng): http://ctan.uib.no/info/latex2e-help-texinfo/latex2e.html

% The document class defines the type of document. Presentation, article, letter, etc. 
\documentclass[12pt, a4paper]{article}

% packages to be used. needed to use images and such things. 
\usepackage[pdfborder=0 0 0]{hyperref}
\usepackage[utf8]{inputenc}
\usepackage[english]{babel}
\usepackage{graphicx}
\PassOptionsToPackage{hyphens}{url}

% hides the section numbering. 
\setcounter{secnumdepth}{-1}

% Graphics/image lications and extensions. 
\DeclareGraphicsExtensions{.pdf, .png, .jpg, .jpeg}
\graphicspath{{./images/}}

% Title or header for the document. 
\title{
	Ledelse i Praksis, Øving 3. Konflikthåndtering. 
}
% Author
\author{
	Magnus L Kirø \\
	IT-sjef, Studentmedien i Trondheim AS. 
}
\date{\today}

\begin{document}
\maketitle
\pagenumbering{arabic}

\section{1, Typiske Konflikter}
 
Typiske konflikter som oppstår i min avdeling er konflikter hvor medarbeideren
ikke gjør jobben sin skikkelig. Dette skyldes typisk dårlig kommunikasjon of
manglende oppfatning av forventninger. For meg personlig er det viktigste å
løse konfliktene med en god tone. Mest for å sikre effektivitet og et godt
arbeidsmiljø. Personlig tar jeg nok litt kjapt avstand til konflikter og ser på
dem som mindre viktig enn de egentlig er. 

\section{2, Unngå konflikter}
Slike standard konflikter ungås enkelt ved gjevnlige medarbeidersamtaler. 
For gruppen som helhet er det viktig at de også tar initiativ. I en frivillig
organisasjon er det ikke bare opp til lederen å få til ting, men hver enkelt må
gjøre det beste for seg. 

Tiden det kan ta fra noe oppfattes som en konflikt og noen tar tak i konflikten
vil variere veldig. Dette har mye med prioriteringer å gjøre. Noen vil
prioritere å løse personalkonflikter fort og ordenltig, mens andre vil ha mer
eierskap til organisasjonen og prioritere det først. 

Mine opplevelser går på at mange mindre konflikter bli nedprioritert da de ikke
direkte hindrer driften av selskapet. Dette har mye å gjøre med at andre
oppgaver har høyere hindringskraft på organisasjonen of derfor blir prioritert.

Her bør lederen veie opp de fordeler og ulempler som kommer med en
nedprioritering av en medarbeider. Medarbeidere blir gjerne veldig lite
effektiv. Noe som kunne vært en ressurs ellers.

\section{3, Håndtering av konflikter}
Når situasjonen har oppstått og man tar tak i det er det bare å hoppe ut i det.
Ta personen til side, avtal et møte e.l. Og snakk om saken. Perosnlig ville jeg
sendt epost først og hørt hva som var greia, og så innkalt til møte for å
diskutere saken og komme til en god løsning. 

I en slik samtale vill ejeg involvert de partene som er aktuelle i konflikten.
Jeg ville tatt styringa i samtalen og hatt en strukturell ledelse på det hele.
Det er viktig å ha kontroll på samtalen og forholdene rundt, slik at alle
parter kommer godt ut av konflikten. 

Det er viktig å være forberedt til slike samtaler, så jeg ville forberedt meg
godt først. Tatt med notater, planlagte spørsmål og en tentativ agenda.   

\section{4, Uakseptable Konflikter}
Ved konflikter på uakseptable nivåer benytter man seg av den autoriteten man
har og kaller partene inn på tepppet med en gang. Da skjønner alle parter
alvoret og man kan ta seg litt tid til å roe seg ned. Det er viktig å holde
hodet kaldt når man har kommet til en situasjon der konflikten er uakseptabel. 

Vidre ville jeg hørt begge parters side og trukket noen rettningslinjer for
vidre adferd og samhandling intil konflikten er løst. 

Om en eller flere av partene ikke skjønner alvoret i saken må man bruke
sterkere virkemidler. Slik som utestengning og permisjon. Eventuelt kalle inn
sin egen leder for å vise hvor alvorlig det er. Man må heller ikke være red for
å søke hjelp hos andre for å få løst konflikten.  

Min opptreden i en slik sammenheng sette presedensen for vidre adferd i
arbeidsmiljøet. Blir regler brutt av en vil de bli glatt ignorert senere. Det
er viktig å sette ned foten og vise at lederen, som man er, har makt og
selvtillit nok til å si ifra om ting når noe ikke er greit. 

Det kan også være greit å si ifra til gruppen som helhet at det fins en
konflikt og be dem vise hensyn til den. I slike vanskelige konflikter er det
også viktig å tenke på at hele arbeidsmiljøet likder, ikke bare de direkte
involverte partene.   

\section{5, Konflikter rundt meg}
De konfliktene jeg har håndtert og observert har vært mindre og ikke direkte
innvolvert med meg, så det har vært veldig greit å forholde seg til.
De fleste konflikter får er greit utfall. Om ikke grei, så ihvertfall rikgit. 

Konfliktene jeg har opplevd i senere tid har mye vært på grunn av dårlig
kommunikasjon på tvers i organisajonen. En del informasjon som noen tror man
ikke trenger å dele kommer på avveie og andre føler at det er blitt satt
utenfor eller ignorert. Det har også hendt at man kunne avverget hendelser om
man hadde all informasjon som var tilgjengelig på saken det gjalt. 

Håndteringen av konflikten syns jeg var dårlig og satte alle parter i et dårlig
lys. Ettervirkninger av den dårlig behandlingen av saken har også blitt
oppservert senere. Noe som ikke burde skjedd i det hele tatt.  

\section{6, Løsnings muligheter}
I mange av konfliktene i Studentmediene må man bli sett først, og så vente på
sin mulighet for å komme med konstruktive innspill. Dette resulterer ofte i
trege prosesser. Noe som gjør at man i mange tilfeller tar avgjørelser før man
tar diskusjoner i plenum. 

De konflikter som oppstår må man evaluere før man begynner med mekling. Det er
viktig å finne partene til konflikten og bakgrunnen for den før man tar affære.
Det er også veldig viktig å være forberedt til konfliktløsningen. Ansienitet og
opparbeidet tillit er også viktig for å gjennomføre vellykkede maklinger. 

Sjøvold snakker mye om SPGR rommet og polariseringer i grupper. Disse
polariseringene er det viktig at man blir god på å se og oppfatte. Ledelse
krever trening.  

Kapittelet i seg selv gir ikke direkte ideer til konfliktløsning. Dette har mye
å gjøre med at det fins så mange forskjellige konflikttyper og man må tilpasse
seg enhver konflikt på litt forskjellig måte. Det kapittelet derrimot gjør er å
komme med en del gode prinsipper for konflikthåndtering og en del faresignaler
man skal se etter. 

\end{document}
This is never printed
