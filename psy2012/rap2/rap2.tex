% latex article template

% cheat sheet(eng): http://www.pvv.ntnu.no/~walle/latex/dokumentasjon/latexsheet.pdf
% cheat sheet2(eng): http://www.pvv.ntnu.no/~walle/latex/dokumentasjon/LaTeX-cheat-sheet.pdf
% reference manual(eng): http://ctan.uib.no/info/latex2e-help-texinfo/latex2e.html

% The document class defines the type of document. Presentation, article, letter, etc. 
\documentclass[12pt, a4paper]{article}

% packages to be used. needed to use images and such things. 
\usepackage[pdfborder={0 0 0},colorlinks=true,linkcolor=blue]{hyperref}
\usepackage[utf8]{inputenc}
\usepackage[english]{babel}
\usepackage{graphicx}
\usepackage[top=3cm, bottom=3cm, left=2cm, right=2cm]{geometry}
\PassOptionsToPackage{hyphens}{url}
\usepackage{setspace}

% hides the section numbering. 
\setcounter{secnumdepth}{-1}

% Graphics/image lications and extensions. 
\DeclareGraphicsExtensions{.pdf, .png, .jpg, .jpeg}
\graphicspath{{./images/}}

% Title or header for the document. 
\title{
	Infant speech perception
}
% Author
\author{
%	Magnus L. Kirø \\
}
\date{\today}

\begin{document}
\onehalfspace
\maketitle
\pagenumbering{arabic}

Course code: PSY2012

Semester: Autumn 2014

Candidate number: 10077

Course responsible: Ruud van der Weel

\tableofcontents
\newpage 

%    Infant speech perception

%max 5 pages. aim for 4.  
\section{Introduction}

%What is speech
"The faculty or act of expressing or describing thoughts, feelings, or
perceptions by the articulation of words." \cite{dspeech}. For infants it is
the same thing, only that infants are very bad at it. So speech for infants are
regarded as its own research area, where the perception of speech is important.
A main focus is the development of perception over time, during the first six
of twelve months of infancy.  

%Introduce a problem. why is it important?
%What is the state of the art about infants atm?
This essay will look into a tiny sample of the latest articles available. That
will give insight into the current research of infant speech and its perception. 
Overview of the current state of research, in a specific field of research, is
important due to the fact that a research field is moving fast. New discoveries
and inventions comes faster and faster, and it will be harder over time to keep
up with current events. 

%Articles chosen
A quick article search resulted in many resent articles, but few directly
related to speech in infants, and perception of it. But a selection was found
of sciencedirect.com. The four chosen articles present work in the field, more
specifically about infants perception of fluent speech. 

%Present the chosen articles 
Lewkowicz et al. (2015) describes multisensory audiovisual speech and infants
perception of it \cite{fluentAVspeech}.   
McMurray et al. (2013) talks about the development of speech perception and
infant directed speech \cite{idsdev}. 
Pins and Lewkowicz (2014) has a different angle, and looks at speech synchrony
and unfamiliar fluent speech \cite{speechsynchrony}. 
Kubicek et al. (2014) looks at the intersensory perception of fluent
speech \cite{intersensory}. 

\paragraph{Outline}
%the order of what will be discussed later. 
Continuing the essay a brief description of speech development and language
perception will be followed by the presentation of, the previously mentioned, 
four articles and their contribution to the research field. Lastly there will
be a concluding section that sums up the findings of the articles and the
contribution of this essay.  

\section{Infant speech}
%present a discussion about the articles. Shortcomings?
For factual basis we should know what speech development of infants are, and
how infants develop their perception of language in the first twelve months.
This is a brief description, in a more extensive essay this would be elaborated
further.

\paragraph{Speech development}
foetuses can process sounds of mothers speech, and distinguish language and
other sounds. Newborns prefer their mothers muffled voice. Prefer mothers voice over others.
Also prefer native language over others. In the first months babies learn how to differentiate rhythmic groups of
languages; stress-time, syllable-timed, and mora-timed. \cite{lecture} 

\paragraph{Language perception}
develops quickly in the first 6 months. In that period infants learn how to
distinguish vowels, and their perception start to reflect contrasts in the
mother tongue \cite{lecture}. Distinction of non-native contrasts are no longer
easy for children after 10-12 months. Different categories are also perceived to a varying degree.
After 6 months infants find it harder to discriminate prototypes.

Consonants are easier to distinguish than vowels. Vowels and consonants are
easily distinguished. Manners of articulation are
also discriminated within the first 6 months. Voicing can also be a problem for
infants. Auditory and visual information if perceived by infants, but not
always integrated together. \cite{lecture} 

\paragraph{Fluent audiovisual speech in infancy.}
The paper of Lewkowicz et al.(2015) investigates infants of 4-, 8- to 10-, and
12- to 14-month old English learning infants. Emergence of mulitisensory
perception and development of its coherence to audiovisual fluent speech is 
investigated.\cite{fluentAVspeech}

Four experiments create the base of the discussion in the paper.

The first experiment tests if 4-month-old infant can match synchronous audible
and visible speed streams \cite{fluentAVspeech}. Infants focus on the face is
used as a trigger mechanism for multisensory coherence. It is assumed that
infants focus more on the face of a talking person when they try to learn how to combine, or
synchronise, auditory and visual input.  

The second experiment is the same as the first one, except this time 8- to
10-month-old infants are used. It extends the from the results of the first
experiment, where it was indicated that 4-month-old infants does not perceive
in a coherent multisensory way. The age range for experiment two was chosen
because it is around this age that infants begin to attend specifically to
audiovisual speech.  

Experiment three takes the negative results of the first two experiments and
changes the age factor. This time 12- to 14-month-old infants are used in the
experiment. And the paper expected that by this age infants would perceive
multiseonsory speech coherence. \cite{fluentAVspeech} 

The fourth experiment focuses on the language aspect. Experiment three shows that
12- to 14-month-old infants successfully detect coherence with both Spanish and
English. To test the language aspect the audiovisual material was
desynchronised, and experiment 3 was repeated with the new data. 

Resulting from the four experiments it is shown that 12- to 14-month-old
infants do not depend on the visual and audible speech streams being
synchronised, while they evidence of multisensory matching.   

\paragraph{IDS, development of speech perception.}
IDS, infant directed speech, "is a speech register characterized by simple
sentences, a slower rate, and more variable prosody"\cite{idsdev}. The paper
looks into recent developments in IDS, and focus on a new cue, Voice onset time
among other things. 

The article sets out to three questions: 'Does IDS improve perception and
development?', 'Does the effect of IDS extend to new contrasts?', and 'Are IDS
effects independent if more basic changes?'\cite{idsdev}. These questions are
elaborated in an experiment. The participants of the experiment was English
speaking Caucasians in parent-infant dyads. Infants aged from 9 to 13 months.  

In general terms the article finds that segmental cues associated with IDS
might be a by-product of a different prosodic structure and slower rate of
speech. This effects the statistical learning of speech in infants. 

\paragraph{Speech synchrony.}
The article of Pons and Lewkowicz (2014) looks at the effect of linguistic
experience and language familiarity, with its relation to audio-visual
synchrony in fluent speech. Delay of 366, 500, and 666 ms are used on the video
streams in the experiments. Spanish and Catalan was used in different groups in the first
experiment. While English was used with the same groups as before in the second
experiment. \cite{speechsynchrony}

The experiments resulted in both groups detecting a 500, and 666 ms asynchrony.
The language familiarity also consistent with the, usually observed, perceptual
tuning in infant response to linguistic input. Concluding the article says that
'there is little doubt that the complexity of fluent
speech is likely to interact with older infants' increased efficiency and
expertise fir processing audiovisual fluent speech'\cite{speechsynchrony}. 

\paragraph{Intersensory perception of fluent speech.}
Kubicek et al. (2014) checks whether or not IDS facilitates intersensory
matching of audio-visual fluent speech. This is checked with 12-month-old
German-learning infants, using German and French fluent speech. 

The paper has two experiments, one where sentences was voiced in adult directed
manner, and the second where sentences was pronounced in IDS manner. In the
first experiment 12-month-old infants did not exhibit matching of visual and
auditory input. In the second experiment infants did perceive relations between
visual and auditory information. So the article concludes that IDS might
influence intersensory perception of fluent speech. 

\section{Conclusions}\label{conclusions}
%We worked hard, and achieved very little.

%Summarise and conclude. Come to some kind of understanding.
%What can be drawn out of what you just presented?

As discussed in the previous section we can see that fluent speech is an
interesting trend in the research field of infant speech perception.
The consensus is that fluent speech has its effects on infant speech
perception. I would argue that fluent speech has an increased learning effect, but only as
a natural extension of the early stages of infant speech perception in the age
up to around 12 months. 

The different experiments mentions across the articles presented shows that
audio-visual stimuli has a definite effect on perception. But that it is not
until after 10 or so months that infants start to process this kind of sensory
input in combination. Before this stage infants mostly focus on sounds, vowels
and consonants. While afterwords full words and later sentences, rather
pictures of words not sounds, become the focus.    
 
State of the art wise, we can see that fluent speech in infant speech
perception is trending, and that the different approaches look similar
approached of research. A common denominator is the multisensory input and what
effects it has on speech perception of infants. 


% bliography
%\cite{deyer}
\addcontentsline{toc}{section}{References}
\begin{thebibliography}{1}
%\bibitem{deyer}
%Dyer, J. H., Kale, P., & Singh, H. (2001). Strategic alliances work. MIT Sloan
%Management Review, 37-43.

\bibitem{lecture}
Kjellrun Englund, Lecture slides, Cognitive Psychology, PSY2012, NTNU, autumn
2014. 

\bibitem{dspeech}
Definition of 'speech', from the free dictionary, 28.011.2014: http://www.thefreedictionary.com/speech

\bibitem{fluentAVspeech}
David J. Lewkowicz, Nicholas J. Minar, Amy H. Tift, Melissa Brandon. Perception
of the multisensory coherence of fluent audiovisual speech in infacy: Its
emergence and the role of experience. Journal of Experimental Child Psychology,
130 (2015), 147-162. 

\bibitem{idsdev}
Bob McMurray, Kristine A Kovack-Lesh, Dresden Goodwin, William McEchron. Infant
direct speech and the development of speech perception: Enhancing development
or an unintended consequence? Cognition 129 (2013) 362-378.

\bibitem{speechsynchrony}
Ferran Pins, David J. Lewkowicz. Infant perception of audio-visual speech
synchrony in familiar and unfamiliar fluent speech. Acta Psychologica 149
(2014) 142-147. 

\bibitem{intersensory}
Claudia Kubicek, Judit Gervain, Anne Hillairet de Boisferon, Oliver Pascalis,
Hélène Lævenbruck, Gudrun Schwarzer. 
The influence of infant-direct speech on 12-month-olds' intersensory perception
of fluent speech. 
Infant Behavior \& Development 37 (2014) 644-651.

\end{thebibliography}


\end{document}
This is never printed
