% latex article template

% cheat sheet(eng): http://www.pvv.ntnu.no/~walle/latex/dokumentasjon/latexsheet.pdf
% cheat sheet2(eng): http://www.pvv.ntnu.no/~walle/latex/dokumentasjon/LaTeX-cheat-sheet.pdf
% reference manual(eng): http://ctan.uib.no/info/latex2e-help-texinfo/latex2e.html

% The document class defines the type of document. Presentation, article, letter, etc. 
\documentclass[12pt, a4paper]{article}

% packages to be used. needed to use images and such things. 
\usepackage[pdfborder={0 0 0},colorlinks=true,linkcolor=blue]{hyperref}
\usepackage[utf8]{inputenc}
\usepackage[english]{babel}
\usepackage{graphicx}
\usepackage[top=3cm, bottom=3cm, left=2cm, right=2cm]{geometry}
\PassOptionsToPackage{hyphens}{url}
\usepackage{setspace}

% hides the section numbering. 
\setcounter{secnumdepth}{-1}

% Graphics/image lications and extensions. 
\DeclareGraphicsExtensions{.pdf, .png, .jpg, .jpeg}
\graphicspath{{./images/}}

% Title or header for the document. 
\title{
	Perception 
}
% Author
\author{
%	Magnus L. Kirø \\
}
\date{\today}

\begin{document}
\onehalfspace
\maketitle
\pagenumbering{arabic}

Course code: PSY2012

Semester: Autumn 2014

Candidate number: 10077 

Course responsible: Ruud van der Weel

%State clearly what angle I have and why I have chosen it. 

\tableofcontents
\newpage 

\section{Introduction}

"Perception is the organization, identification, and interpretation of sensory information in
order to represent and understand the environment."\cite{wiki}. 
The five senses represent the main organs of perception. These, and theories
about how they work are described in "Visual Perception: Physiology, psychology
and ecology"\cite{visualbook}

\paragraph{The History}
of perception has evolved throughout the ages. From the idea that the eyes
contact the world, to the world contacts the sensory organs, and further to the
belief that organs and the world are united through a medium \cite{lecture}.

The understanding of perception in psychology has advanced by combining
different techniques, following the rise of experimental psychology.  
Psychophysic, sensory neuroscience, and perceptual issues in philosophy are
areas of current experimentation.   

The discussion of direct or indirect perception has evolved from the
traditional belief that perception was passive, to stronger belief in that
perception is active. The debate is still going over the extent of what is
active and what is passive perception. As an example, some think about vision
as matrix of sensory information that is interpreted in the brain. Then, is the
perception passive, and the interpretation active?    

\paragraph{There are four main theories of perception,} perception as direct
perception, perception-in-action, evolutionary psychology and perception, and theories 
of visual perception. 

Perception as direct perception is the theory that claim sensations cannot
provide a unique world description alone. A mental model is required to enrich
the sensations. Gibson has a different view of direct perception, the approach of
perceptual ecology. He says information is presented to a perceptual system.
\cite{wiki}


Perception-in-action is derived from Gibson's early work. The core idea is that
action would be unguided without perception, and perception would be
purposeless without action. Two distinctions in this main theory are
about invariants. Gibson thinks that assumptions are singular entities that exist in the world and
perception only focus on them, while constructivism, a theory held by
Glasersfeld\cite{wiki}, says that actions on the external input, and
continuous adjustment of perception is the entity.  

Evolutionary psychology on perception thinks that perception is mostly for
collision avoidance, or moving around in space. They say that the need to
decide the distance to other objects has helped evolve perception. 
Modularity is central to perception in evolutionary psychology. There exists modules for all
the sense. The modules are specialised sections of the brain that handles one
particular input, or perception tasks. 

Theories of visual perception are many. Some of them are:  Empirical theories of
perception, Enactivism, Anne Treisman's feature integration theory, Interactive
activation and competition, and Irving Biederman's recognition by components
theory. \cite{wiki} 

\paragraph{Outline}
%How this essay enlightens interesting parts of perception
%art1, 2, 3
%visual perception. 

The base of the following part are three articles: When perception says "no" to action: Approach cues make
steep hills appear even steeper \cite{noaction}, Visual illusions and direct perception:
elaborating on Gibson's insights \cite{visualillusion}, People
perception: Social vision of groups and consequences for organizing and
interacting
\cite{peopleperception}. Insight into these articles will be provided, and
discussed briefly.

\section{Three aspects of perception in psychology}\label{discussion}
A small selection of articles, three, are presented here as an introduction to
perception. To show the diversity in the field of research and present current
idea are key elements of this essay. 

\paragraph{When perception says "no" to action.}
Krpan and Schnall (2014)\cite{noaction} experiments with the perceived heigh of a hill. They
have three experiments where people estimate the height of the hill. The base
assumption is that peoples resources, previous knowledge and capabilities plays
an important role in visual perception. In essence the experiments test whether
or not perception is effected by impending actions. 

The three experiments test different cues for visual perception. Experiment 1
tests if arm flexing, and arm extension, as cues has any effects on the perception
of a steep hill. Experiment 2 builds on experiment 1, and test people in good
physical condition. Different cues are tested in experiment 2, such as the
absence of action or inaction cues, and those compared to avoidance cues.
In experiment 3 climbing of the hill is implicit. The point of this is to
investigate the mechanisms of the previous two experiments. 

Experiment 1 and 2 resulted in steeper hill estimations for approach induced
movement. Physically conditioned participants showed lower estimations, as they
are more likely to undertake actions with higher physical requirements. 
Experiment 3 shows that people report higher climbing propensity when
approaching and are told to climb the hill.  

In general the article shows that peoples experience and knowledge of personal
skills affect perception. 

\paragraph{Visual illusions and direct perception.}
The article elaborates on Gibsons treatment of geometrical illusions, and looks
into empirical evidence of it \cite{visualillusion}. Gibsons insights has been
corroborated, but the general Gibsonian principles of perception needs to be
adjusted in accordance to the empirical data. The article builds on the
reconceptualization of Withagen and Chemero(2009)\cite{visualillusion}, about
the use of information and observer variability. Different illusion effects in
terms of the detected optical variables are visualised in the developed
ecological approach. 

Gibsons direct perception theory says that the detection of information is the
only thing required to perceive the environment, given a one-to-one relation
between optical variable information and the environment. Several conceptual
problems are dissolved with the indirect approach by this. Defining information
as specification arises the problem of how could Gibson account for instances
of illusions?

M.M. de Wit et al. says that empirical literature have cast doubt on Gibsons
conceptualisation of information as specificity in a one-to-one relation in
perception \cite{visualillusion}. The article enlightens the reader on the aspects of Gibsons accounts of
illusions, and corroborates them. Yet, the detection of specifying information
as a concept of perception, the Gibsonian concept, is punt into doubt. 
Concluding, the article states that a non-specifying optical, correlating to
future perceivable property, variable results
in susceptibility to visual illusions. 

\paragraph{People perception}
Perception as a tool to interpret group dynamics and relations is an interesting
perspective of perception as a concept. The importance of cognitive processes when
evaluating groups have been confirmed by ample research
\cite{peopleperception}. Literature about perception of the individual create
the foundation of perception of groups. Perception of groups are especially
useful for managers and leaders, and is interesting to explore in an
organisational context. 

Perception can, alternatively to the traditional view, be the "process of
forming and interactive with mental representations about people"
\cite{peopleperception}. The traditional view is that perception refers to the
immediate sensory input from touch, hearing, sight, taste or smell.

While different definitions of 'group' exists, a common denominator is that it
is a collection of people. Moreover, groups have to be perceivable in whole.
The observer has to see the whole group at the same time. By this definition,
'women' and 'Apple' are not groups. 

The paper proposes a model based on selection, extraction and application, the
SEA model \cite{peopleperception}. They use the model when discussing
organizational implications. Rapid, and automatic, perceptions of groups are
important to psychological phenomena. Perception of groups are also important
in judgements and decision-making \cite{peopleperception}. Much literature has
researched the deliberative and automatic process of group development.  

Phillips sums up the paper by stating that "perceiving groups is critical for
organizational and social functioning" \cite{peopleperception}. Understanding
of, impressions of, and perceptions of groups are stated by many organizational
and social theories as central processes in social and organizational life. 
Processes and implications involving visual perception of groups are the
subject, of which the general idea about people perception, should be explored. 
Essential organizational and social dimensions are suggested to be rapidly,
automatically, and correctly represented through visual perception of groups. 

\section{Conclusions}\label{conclusions}

This essay draws on resent literature to give a peak inside the field of
perception in psychology. Three articles were presented, and some brief
historical points pointed out.

First an experimental aspect of perception was elaborated. It showed three
experiments the build on one another. From the perception of steepness of a
hill based on different cues. Then an extension to consider fit people and
thereby biological properties of self insight. Last of the experiments was the
addition of intent, or rather the known action to be undertaken, and the
intents effect on perception. 

Secondly visual illusions, and Gibsons view of direct perception was enlightened. A dive into
important theory and realisation of critical thinking was touched when de
Wit\cite{visualillusion} stated that Gibsons theories must be updated to better
fit reality. 

Thirdly the concept of visual perception in a group setting was presented. How
perception in psychology can be used to research new fields of perception
becomes apparent. While perception is set in an everyday use case.
\cite{peopleperception} 

Experiments, theories, and practical use sums up the papers main parts.
The three parts presents important aspects of perception in psychology.

% references / bibliography
\addcontentsline{toc}{section}{References}
\begin{thebibliography}{1}
%\bibitem{deyer}
%Dyer, J. H., Kale, P., & Singh, H. (2001). Strategic alliances work. MIT Sloan
%Management Review, 37-43.

\bibitem{lecture}
Ruud van der Weel; Lecture slides, Cognitive Psychology - PSY2012, NTNU, autum
2014.

\bibitem{visualbook}
Vicki Bruce, Patrick R. Green, and Mark A. Georgeson; Visual Perception,
Physiology, psychology and ecology, 4th edition, Psychology Press, 2003.

\bibitem{wiki}
Wikipedia on Perception, https://en.wikipedia.org/wiki/Perception, 25.11.2014.

%\bibitem{visualmasking}
%Visual Masking: Studying Perception, Attention, and Consciousness, Talis
%Bachmann, Gregory Francis, Visual Masking, 2014, Pages 1-108.

\bibitem{noaction}
When perception says "no" to action: Approach cues make steep hills appear
even steeper, Darion Krpan and Simone Schnall, Journal of Experimental Social
Psychology, 55, 2014, p89-98.

\bibitem{visualillusion}
Visual illusions and direct perception: Elaborating on Gibson's insights,
Matthieu M. de Wit, John van der Kamp and Rob Withagen, New Ideas in
Psychology, 36, p1-9. 2014.

\bibitem{peopleperception}
Phillips, L. T., et al. People perception: Social vision of groups and consequences for organizing and
interacting. Research in Organizational Behaviour (2014).
http://dx.doi.org/10.1016/j.riob.2014.10.001 

\end{thebibliography}

\end{document}
This is never printed
