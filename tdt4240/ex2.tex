% latex article template

% cheat sheet(eng): http://www.pvv.ntnu.no/~walle/latex/dokumentasjon/latexsheet.pdf
% cheat sheet2(eng): http://www.pvv.ntnu.no/~walle/latex/dokumentasjon/LaTeX-cheat-sheet.pdf
% reference manual(eng): http://ctan.uib.no/info/latex2e-help-texinfo/latex2e.html

\documentclass[12pt, a4paper]{article}
\title{Exercise TDT4240: Patterns}

\PassOptionsToPackage{hyphens}{url}
\usepackage[pdfborder=0 0 0]{hyperref}
\usepackage[utf8]{inputenc}
\usepackage[english]{babel}
\usepackage{graphicx}

% hides the section numbering. 
\setcounter{secnumdepth}{-1}

% Graphics/image lications and extensions. 
\DeclareGraphicsExtensions{.pdf, .png, .jpg, .jpeg}
\graphicspath{{./images/}}

% Author
\author{
        Magnus L Kirø \\
		Håvard Tørresen
}
\date{\today}

\begin{document}
\maketitle
\pagenumbering{arabic}

\subsection{Step 4 - theory}

\subsection{a} 

Design patterns: Template method, Observer, Abstract factory, State, Pipeline
Architectural patterns: MVC


Architecture is the overall plan. The bigger picture. The patterns here describe high level goals and intentions. 

Design is more speciffic. It is the small picture. The implementation specifics are decided in this phase. 

\subsection{b} %done 
We chose the Observer pattern. The class that was changed is the Collision class, inside the PongScreen class. 
The ballSprite class works as the subject, while the collision class has the role of the observer.  

\subsection{c} %done
Yes, there are advantages of using the Observer pattern in this program. 

The main advantage of this is that we have controll of all the collisions in one class. 
Another advantage is that the implementation is simpler because we don't have to manyally check overlaping points in sprites. 

\subsection{Selfevaluation}
For the total time of this exercise we used aproximatly three hours. 

\end{document}
