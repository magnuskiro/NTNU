% latex article template

% cheat sheet(eng): http://www.pvv.ntnu.no/~walle/latex/dokumentasjon/latexsheet.pdf
% cheat sheet2(eng): http://www.pvv.ntnu.no/~walle/latex/dokumentasjon/LaTeX-cheat-sheet.pdf
% reference manual(eng): http://ctan.uib.no/info/latex2e-help-texinfo/latex2e.html

% The document class defines the type of document. Presentation, article, letter, etc. 
\documentclass[12pt, a4paper]{article}

% packages to be used. needed to use images and such things. 
\usepackage[pdfborder={0 0 0},colorlinks=true,linkcolor=blue]{hyperref}
\usepackage[utf8]{inputenc}
\usepackage[english]{babel}
\usepackage{graphicx}
\PassOptionsToPackage{hyphens}{url}

% hides the section numbering. 
% this makes \ref{marker} show up as empty. use \nameref{}, or \pageref{}
%\setcounter{secnumdepth}{-1}

% Graphics/image lications and extensions. 
\DeclareGraphicsExtensions{.pdf, .png, .jpg, .jpeg}
\graphicspath{{./images/}}

% Title or header for the document. 
\title{
	Sensing and Perception, what is the difference?
}
% Author
\author{
	Magnus L. Kirø \\
}
\date{\today}

\begin{document}
\maketitle
\pagenumbering{arabic}

%\begin{abstract}
%1200 words, deadline 10.03, topic: Explain the difference between sensing and perception.
%\end{abstract}

\section{Introduction}

The understanding of perception has been a topic for discussion a long time.
From the ancient Greeks and through modern times until today. The Greeks
introduced the three perspectives on perception. Something reaching out to
sense the world, something radiating from the world and contacts the sensing
organs, and the thought that the world and the sensing organs are connected
through a medium. 

In modern days we have split the act of perceiving into two aspects. The
sensory system, what happens in the body, and the cognitive aspects, what
happens in the mind. The sensory system is the complex biological combination
of sensors, nerve system, and brain. "Perception is the organization,
identification, and interpretation of sensory information in order to represent
and understand the environment" \cite{Wiki-perception}.

Both the sensory system and perception is affected by external influences. 
Perception is affected by several factors such as learning, memory,
expectations, and attention. These factors shapes our perception of
experiences. The sensory system can be affected by disabilities, or more often
drugs. 

Psychophysics describe relations between perception and sensory input from
physical qualities. An application of psychophysics is lossy-compression of
media content, where concepts of psychophysics have enlightened how humans
perceive sound and vision. Psychophysics is also referred to as a set of
methods that can be used for studying perceptual systems.
\cite{Wiki-psychophysics} 

Other significant developments of perception are events like the cognitive
revolution. Where anthropology, psychology, and linguistics were combined with
the new developments and ideas from computer science, neuroscience, and
artificial intelligence. \cite{Wiki-cognitiveRevolution} 

The body as the sensory system, and the mind as perception is presented next. 
Both as different aspects of the perceiving system, where sensing and
perception are different things. A discussion of the core question, what is the
difference between sensing and perception, will conclude the essay. 

\section{Body}
The five common senses that create the sensory system touch, smell, taste,
hearing and sight. The vestibular system, responsible for balance, are also part
of the sensory system. The information from the sensory system is transmitted to
the brain, where it is interpreted, and transformed into the foundation for our
perception or view of the world around us.  

Stimulus is the central task of the sensory system. It reacts based on the
stimulus that is exerted on the body. The stronger the stimulus, the stronger
the sensory signal, and the stronger the reception in the brain. Along with
intensity duration and location is important in relation to stimulus.  

The sensory system consists of three main parts: the neural pathways that
transmits signals, the sensory receptors, like hands, eyes, and the nose, and
parts of the brain that interprets the electrical signals that represent the
different observations of the sensory inputs.

Essential concepts of the sensory system are receptors, cortex, modality. 

There are five types of receptors in the body: chemo-, photo-, mechano-,
thermoreceptors, and nociceptors. The chemoreceptors detect chemical stimuli.
Tasting buds are an example of chemoreceptors. The photoreceptors detect light,
eyes are photoreceptors. There are four categories of mechaoreceptors that
react to mechanical stimuli of different intensities. Thermoreceptors are
affected by temperature, warm and cold. Nociceptors are internal sensors that
react to potential damage to internal organs.  

Accompanying the senses we have cortices in the brain. Somatosensory cortex for
touch, auditory cortex for processing sound, gaustatory cortex for processing
taste, primal olfactory cortex to interpret smell, and the visual cortex to
handle the optical information from the eyes.

Modality is the physical phenomenon that is sensed by the sensory organs,
taste, smell, pressure, and temperature. 

The information from the different censors, the cortices in the brain, and the
sensory modality of the world creates, in combination, a foundation of
information for perception.

\section{Mind}
Perception can roughly be described in two perspectives, bottoms-up and top-down. 
The top-down approach starts at the highest level of abstraction, and works its
way down. It uses attention, concepts, and knowledge to comprehend the sensory
input. The bottoms-up way is to take the smallest building blocks first and gradually
build a bigger picture. Here low-level information is processed to become
higher-level information. An example of this is that rough shapes can be
interpreted as objects. \cite{Wiki-perception}
Attention, learning, memories, and expectations are part of perception. These
four aspects are what represents the experience part of perception. 

In Gestalt psychology perception focusses on the whole picture. 
”Perception of the whole is prior to that of its parts” - Wertheimer. And 
”The whole is more than the sum of its parts” - Aristotle. 
The world is apparently chaotic in Gestalt psychology. This has its effect on
the ability to maintain and acquire meaningful perceptions. An understanding of
laws in this context is what the Gestalt psychologist try to find. The notion
that the mind creates a self-organizing global whole in this chaotic world is a
central principle. \cite{Wiki-gestalt} 

Direct vs indirect perception. 
A popular discussion in psychology is direct vs indirect perception. The two
thoughts that we on one side perceives the world directly, and the other, that
we interpret the sensory input along the way. Here we have two distinct
thoughts of how mind and body works on perception. In direct perception the
body more or less places the finished though in the mind, while indirect
perception provides the mind with information to be processed. Direct
perception leaves experience and memory to be spice for the idea, while
indirect perception uses memory and experiences as a filter for the sensory
image. \cite{Wiki-naiveRealism} 

Perception-in-action is the theory where action and perception is closely
connected. Action is not possible without perception, and perception serves no
purpose without action. Continual adjustment of perception and action to
external input is regarded in constructivism as what constitutes the 'entity'. 

Evolutionary psychology if associated with perception in some forms. Some
evolutionary psychologists thinks that the purpose of perception is not knowledge,
and that a primary use of perception is to guide movement. It is as an example
suggested that eyesight is used to direct action, collision avoidance, and not for knowledge. 

Perception is by some thought of to be the whole process of comprehending the
environment around us. While others believe that perception is only the part
where previous experience and sensory input is combined to a new abstract
impression. 

\section{Discussion}\label{conclusions}
The basis for a discussion around perception and sensing has been presented
with the sensory system and theories about perception. Where the sensory system provides the
physical aspect of perception and the theories provide the different ideas of
how perception really works in our mind.

The two polarised ideas are that perception is the whole process of perceiving
the environment we inhabit, and that perception is only the interpreting aspect
of combining sensory information and experiences. 

For the first idea, perception as a whole, reduces the difference of sensing
and perception to more or less the same action/event. This is described by
Gibson in his theory of direct-perception. We can also see this in evolutionary
psychology where it is believed that perception is used to guide action. 

On the other hand, with the second idea, perception is used in many cases to
gain knowledge. People can also use reflection and think of new ideas that
change perception, or change how we perceive the environment we currently
occupy.

The two ideas have two different complexity levels. The first idea is much more
complex than the second. It is easier to understand a modular system where the
sensing happens in the sensory system, and perception is the product of
different sources of information(sensory input, and previously experienced
knowledge).  

% references / bibliography
%\cite{deyer}
\addcontentsline{toc}{section}{References}
\begin{thebibliography}{1}

\bibitem{Wiki-sensorySystem}{
Wikipedia on the Sensory System, 09.03.15, \url{https://en.wikipedia.org/wiki/Sensory_system}
}

\bibitem{Wiki-perception}{
Wikipedia on Perception, 09.03.15, \url{https://en.wikipedia.org/wiki/Perception}
}

\bibitem{Wiki-gestalt}{
Wikipedia on Gestalt psychology, 10.03.15, \url{https://en.wikipedia.org/wiki/Gestalt_psychology}
}

\bibitem{Wiki-psychophysics}{
Wikipedia on Psychophysics, 10.03.15, \url{https://en.wikipedia.org/wiki/Psychophysics}
}

\bibitem{Wiki-cognitiveRevolution}{
Wikipedia on Cognitive Revolution, 10.03.15, \url{https://en.wikipedia.org/wiki/Cognitive_revolution}
}

\bibitem{Wiki-naiveRealism}{
Wikipedia on Naïve Realism, 10.03.15, \url{https://en.wikipedia.org/wiki/Na%C3%AFve_realism}
}

\end{thebibliography}

\end{document}
This is never printed
