% latex article template

% cheat sheet(eng): http://www.pvv.ntnu.no/~walle/latex/dokumentasjon/latexsheet.pdf
% cheat sheet2(eng): http://www.pvv.ntnu.no/~walle/latex/dokumentasjon/LaTeX-cheat-sheet.pdf
% reference manual(eng): http://ctan.uib.no/info/latex2e-help-texinfo/latex2e.html

% The document class defines the type of document. Presentation, article, letter, etc. 
\documentclass[12pt, a4paper]{article}

% packages to be used. needed to use images and such things. 
\usepackage[pdfborder=0 0 0]{hyperref}
\usepackage[utf8]{inputenc}
\usepackage[english]{babel}
\usepackage{graphicx}
\PassOptionsToPackage{hyphens}{url}

% hides the section numbering. 
\setcounter{secnumdepth}{-1}

% Graphics/image lications and extensions. 
\DeclareGraphicsExtensions{.pdf, .png, .jpg, .jpeg}
\graphicspath{{./images/}}

% Title or header for the document. 
\title{
	IT3010-Research Methods in IT, Assignment-2
}
% Author
\author{
	Magnus L. Kirø
}
\date{\today}

\begin{document}
\maketitle
\pagenumbering{arabic}

\section{Data collection plan}
The participant observation method was chosen not to interfere with the setting
of the observations. The setting was a meeting with focus on an educational
report. All participants was focused on the feedback, so the main focus lay
there. This also makes sure that the participants of the observations don't
know about the observations beforehand. Which makes the data less affected by
external parameters. 

How you plan to carry on the observation
I plan to participate normally in the meeting that is going to take place.
Under the meeting I will notice mobile activity in the room, and observe what
purposes the mobile devices are used for. Afterwords I will write about the observations, and my findings.  

\section{Data collection instrument}
%An observation protocol that provides a list of aspects to consider
%during the observation (maximum one A4 page or 450 words)

The main aspects to consider while making the observations are: 
1-what is the device used for?, and 2-what educational purpose does this action
have?. 

The observations are recorded after the meeting, due to the fact that I
personally participates in the meeting.

\section{Observation notes}
Observed applications:
Calendar, sms, bluetooth file transfer, voice recorder, mail, and text
editor(notes).

Emotions: disappointment, regret, happiness, satisfaction.

Communication: Questions, not with devices.  

Interaction: sharing info, gathering info, storing info, taking notes.

\section{Research data}

Observations were gathered on October 1st, 10:00-10:30. 

%setting (external environment, students, social media) 
The observations was gathered during a meeting with six people. The purpose of
the meeting was to gather feedback on a scientific report. This was done
undisturbed by the observations.  

%and detail description of what you saw/ hear 
During the observations the most notable emotion was disappointment. Other
individual emotions was regret, happiness, and satisfaction.

The interactions of the duration were varied. With devices storage of
information was the most common one. Next was the information gathering,
looking stuff up and fact checking. Sharing was also an observed interaction.

Communication during the meeting was mostly one directional. But the important
parts of it was multi directional between many parties. Mostly questions, but
also statements of action and understanding.

The most notable part of the observations was that the use of devices was so
natural it melted into the background. The disturbance of devices in the room
was minimal, and the focus of the meeting was upheld.

%your interpretation of what you saw
Delayed writing of notes and observations might be a factor in the reflection
of the data. But it is not likely to provide give big problems for later work
with the data. 

\section{Reflection}
presents your own reflection on the data collection process,

The data collection method has it's strengths and weaknesses. Strengths starts
with the ease of collecting information. The participants of the observations
do not need to know about the observations taking place. Which renders the data
unbiased by human interaction from the observation subjects. Another strength
is that the participant can easily get close to the observations and therefore
make better observations.

Weaknesses of the collection method are mainly the fact that the observations
cannot be recorded right away. The data has to be written down after the
observation session. This makes the data biased by the observer, and can in bad
cases be distorted from the actual observing until it is written down. Further
for the weakness, the subjects cannot come with input. The input of the
observation subjects can sometimes be useful for further collection of data and
the data integrity. Focus of the observer might also be a problem. Should the
observer focus only on the observation task, of actively partake in the
interactions of the setting? This is a difficult balance. 

During the actual observing, it was rather hard to concentrate on two things at
the same time. The observations mainly based on memories from the meeting. So
all the actual observation notes are a bit stale already. Another unexpected
part was how fast the meeting, and observing, passed. The actual time to
observe feels very short.  

Think about how observing without taking notes affects what you see.
Making observations without taking notes is a good way to pay better attention
to what is going on. While at the same time being worried about remembering all
the observations until they can be recorded. Not taking notes has a negative
effect on the correctness of the observations. But the main aspects of the, or most
notable, observations were remembered and recorded. 

During this assignment I learned that active participation while making
observations is a good method for  collecting data. The method engages the
observer to take a bigger part, and invest in the actual observations. This is
a good thing. Although we have to be careful about when we want to use this
collection technique. The method has to be fitting for the observations we
want, and the setting we have available for data gathering. 

\section{Research Questions}
Below are the research quotations, discussed briefly in correlation with the
data. If we fit the data to the research questions we get the following
discussion.  

\paragraph{In which circumstance student use mobile apps for educational
purpose?}
For the educational purposes students used apps indirectly. Meaning that the
use of apps was not a direct learning tool, but rather an aid to improve the
learning. This can be seen in the apps used during the meeting. This particular
circumstance provides little learning potential to apps.   

\paragraph{What kind of educational activities has been done with mobile apps?}
During the observations there was no direct educational activity with mobile
apps. But apps was widely used to aid the educational process.

\paragraph{What kind of impacts does mobile apps has on student academic life?}
Mobile apps and devices has a huge impact on the academic life. Now there are
room for more focus on the academical parts, as organizational ones have been
removed. But there are also more potential disturbing factors in the mobile
devices. Despite the disturbance potential apps was not a disturbing factor
during the observations. 

\end{document}
This is never printed
