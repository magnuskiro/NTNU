% latex article template

% cheat sheet(eng): http://www.pvv.ntnu.no/~walle/latex/dokumentasjon/latexsheet.pdf
% cheat sheet2(eng): http://www.pvv.ntnu.no/~walle/latex/dokumentasjon/LaTeX-cheat-sheet.pdf
% reference manual(eng): http://ctan.uib.no/info/latex2e-help-texinfo/latex2e.html

% The document class defines the type of document. Presentation, article, letter, etc. 
\documentclass[12pt, a4paper]{article}

% packages to be used. needed to use images and such things. 
\usepackage[pdfborder=0 0 0]{hyperref}
\usepackage[utf8]{inputenc}
\usepackage[english]{babel}
\usepackage{graphicx}
\PassOptionsToPackage{hyphens}{url}

% hides the section numbering. 
\setcounter{secnumdepth}{-1}

% Graphics/image lications and extensions. 
\DeclareGraphicsExtensions{.pdf, .png, .jpg, .jpeg}
\graphicspath{{./images/}}

% Title or header for the document. 
\title{TDT4171 Artficial Intelligence Methods, Exercise 2}
% Author
\author{
        Magnus L Kirø \\
}
\date{\today}

\begin{document}
\maketitle
\pagenumbering{arabic}

\section{Part A}

\begin{itemize}
	\item The unobservable variables for the umbrella example is the weather. The weather is unobservable locally. For each time t we have a waether observation. 
	\item The observable variables are the evidence for the model. For a given time slice t, do we see the umbrella or not? We base our calculations on this observations.

	\item{
		The dynamic model would be like this: \\ 
		\begin{equation}
		P(X_t | X_{t-1}) = 
		\begin{tabular}{ a b }
		  0.7 & 0.3 \\
		  0.3 & 0.7 \\
		\end{tabular}
		\end{equation}
		\\
		While the observation model would be like this: \\ 
		\begin{equation}			
		P(E_t | X_t) = 
        \begin{tabular}{ a b }
          0.9 & 0.0 \\
          0.0 & 0.2 \\
        \end{tabular}
		\end{equation}
        \\
	}

	\item{
		\textbf{First-order Markov process}:
		This model depends only on the previous observation, and not any other earlier states. In the case of weather this will not work.
		Seasons should be considerd, along with the last two or three days. 
		Depending og location and season rain might have higher probability of occuring multiple days in a row.   
		
		\textbf{Stationary Process}: 
		As the wather is controlled by the laws of nature we can conclude that the process is stationary. 
		While the probability of P(R$_t$\|R$_t_-_1$) is always the same, the previous statement is proved. 

		\textbf{Sensory Markov assumptions}:
		Since we base our sensory model on the values observed in a time slice t, we can assume this to be a reasonable conclusion. 

		The assumptions are reasonable for this model.  
	}
\end{itemize}

\subsection{Part B}

Running the forward operation we get this output:

\begin{verbatim}
kiro@kiro ~/ntnu/tdt4171/ex2 $ python ex2.py 
Forward:
fw msg 1: [ 0.81818182  0.18181818]
fw msg 2: [ 0.88335704  0.11664296]
fw msg 3: [ 0.19066794  0.80933206]
fw msg 4: [ 0.730794  0.269206]
fw msg 5: [ 0.86733889  0.13266111]
\end{verbatim}

From this we can see that:

\begin{equation}
P(X_2|e_{1:2}) = 0.88335704
\end{equation}

and given the evidence: \\
{ e1:5 = Umbrella1 = true, Umbrella2 =true, Umbrella3 = false, Umbrella4 = true, Umbrella5 = true } \\
that: 

\begin{equation}
P(X_5|e_{1:5}) = 0.86733889
\end{equation}

\section{Part C}
Running the code we get this output: 

\begin{verbatim}
kiro@kiro ~/ntnu/tdt4171/ex2 $ python ex2.py 
Forward:
fw msg 1: [ 0.81818182  0.18181818]
fw msg 2: [ 0.88335704  0.11664296]
\end{verbatim}

From this we can see that: 

\begin{equation}
P(X_1|e_{1:2}) = [ 0.883,0.117 ]
\end{equation}

Further we have the backward messages:  

\begin{verbatim}
kiro@kiro ~/ntnu/tdt4171/ex2 $ python ex2.py 
(...)
Backward:
bw msg: 5: [1 1]
bw msg: 4: [ 0.69  0.41]
bw msg: 3: [ 0.4593  0.2437]
bw msg: 2: [ 0.090639  0.150251]
bw msg: 1: [ 0.06611763  0.04550767]
bw msg: 0: [ 0.04438457  0.02422283]
\end{verbatim}

\section{Part D}
See ex2.py for the code. 

\end{document}
