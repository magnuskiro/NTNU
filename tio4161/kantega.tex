% latex article template

% cheat sheet(eng): http://www.pvv.ntnu.no/~walle/latex/dokumentasjon/latexsheet.pdf
% cheat sheet2(eng): http://www.pvv.ntnu.no/~walle/latex/dokumentasjon/LaTeX-cheat-sheet.pdf
% reference manual(eng): http://ctan.uib.no/info/latex2e-help-texinfo/latex2e.html

% The document class defines the type of document. Presentation, article, letter, etc. 
\documentclass[12pt, a4paper]{article}

% packages to be used. needed to use images and such things. 
\usepackage[pdfborder={0 0 0},colorlinks=true,linkcolor=blue]{hyperref}
\usepackage[utf8]{inputenc}
\usepackage[norks]{babel}
\usepackage{graphicx}
\PassOptionsToPackage{hyphens}{url}

% hides the section numbering. 
% this makes \ref{marker} show up as empty. use \nameref{}, or \pageref{}
\setcounter{secnumdepth}{-1}

% Graphics/image lications and extensions. 
\DeclareGraphicsExtensions{.pdf, .png, .jpg, .jpeg}
\graphicspath{{./images/}}

% Title or header for the document. 
\title{
	Alternative tolkninger til Kantega-analyse
}
% Author
\author{
	Magnus L. Kirø \\
}
\date{\today}

\begin{document}
\maketitle
\pagenumbering{arabic}

\tableofcontents
\newpage

\section{Introduksjon}
Denne oppgaven går ut på å komme med en alternativ
forståelse, tolkning, eller løsning på tre områder fra gruppe 3 sin analyse av
Kantega. Analysen blir referert til som 'Kantega-analyse' i denne oppgaven. 
Analysen tar for seg prosjekt- og personal-ledelsesperspektiver.

Gruppe 3 har fokusert på rekruttering, og ledelsesutvikling, etter en
overordnet organisasjonsanalyse. Disse aspektene blir også belyst her, men fra
en litt annen vinklig. 

\paragraph{Kantega} er et konsulentselskap, med kontorer i Trondheim og Oslo.
De spessialiserer seg på skredersydde IT-løsninger. De er moderne, og holder
seg oppdatert på de nyeste utviklingsmetodologiene og teknologiene i markedet.
For Kantega er det viktig å oppfylle forventningene til kundene, og utvikle
kompetansen innad i organisasjonen.   

%Beskriver Kantega og empirien rundt. 
Kunnskapsarbeidere, flesteparten av de ansatte,  står i sentrum av den operative
kjernen i Kantega. Kantega ønsker ansatte som vil bruke seg selv, lære noe
ofte, finne ut hvor god vi kan bli, jobbe samen med andre, og bidra til
felleskapet og til samfunnet.

\paragraph{De tre tilnærmingene,} struktur, rekruttering, og ledelse, er de tre
aspektene resten av oppgaven tar for seg. Hovedvekten vil ligge på ledelse, da
det ser ut som det er dette punktet Kantega har lyst til å forbedre mest, eller 
har mest problemer med. 

Gjennom oppgaven kommer det forslag og innspill på følgende tre spørsmål. Kan
strukturen i Kantega endres til det bedre, evt hvordan? Er det egentlig bedre
å rekruttere eksterne ledere, kontra interne? Hvordan kan Kantega forbedre ledelsen?

%Få med teori her! 
Oppgaven tar i hovedsak utgangspunkt i teorien beskrevet av gruppe 3. Dette er
hovedsaklig teori fra Bolman \& Deal om organisasojn og ledelse og Wilton om
personalledelse.

\paragraph{} Oppgavens hoveddel starter med diskusjon om \nameref{struktur}.
Fortsetter med \nameref{rekruttering}, og fullføres med \nameref{ledelse}. 
En oppsummering av diskusjonen og foreslåtte forbedringer kommer sist i
\nameref{konklusjon}.

\section{Struktur}\label{struktur}
%hierarkiet, flatt, men ikke flatt nok?

Strukturanalysen basserer seg på Mintzbergs strukturkonfigurasjoner. 
Som gruppe 3 påpeker er det vanskelig å plassere Kantega i en bestemt
organisasjonsstruktur (Kantega-analyse, gruppe3, s11). Kantega bærer preg av
flere av Mintzbergs
strukturkonfigurasjoner(Bolman og Deal, 2009, s104). Selv mener Kantega at de
er et adhockrati, mens man fra utsiden kan oppfatte Kantega som et fagbyråkrati.
Uoverensstemmelsen mellom de to bildene burde man forbedre på sikt. Et
alternativ er å ta til seg adhockratiet til det fulle og redusere den vertikale
avstanden i Kantega ytterligere.

Den operative kjernen i Kantega består i all hovedsak av utviklings-team.
Teamene har ofte en prosjektleder. Gruppe 3 sier følgende om
prosjektlederbegrepet:  
"Begrepet om en 'prosjektleder' unngås, og man har i stedet en 'Scrum master' som
sørger for det administrative." (Kantega-analyse, gruppe 3, s10)
Ettersom en scrum master har ansvaret for å
holde oversikt over arbeidsoppgaver, beslutningsdokumentering, måling av
framgang, kommunikasjon med ledelse, og kommunikasjon med kunde vil det være
veldig naturlig å kalle rollen 'prosjektleder' i daglig tale (Sommerville,
2011, s72-74). Essensen av å være prosjektleder er å tilrettelegge arbeidet for
teamet best mulig. Med flat struktur er dette et ansvarsområde på lik linje med
resten av utviklerene i teamet. Det er også viktig å påpeke at det ikke er
noe 'ovenfra og ned' perspektiv for en scrum master(Sommerville, 2011, s74).

% dra scrum teamene inn i det at man har flat struktur selv her. Og at man kan
% se på personene som å ha forskjellige ansvarsområder i motsetning til
% forskjellig hierarkisk plassering.  

Siden Kantega allerede i dag har bassert seg på scrum-team vil det være
naturlig å benytte tilhørende kvaliteter i resten av organisajsjonen. 
"Som vi ser er arbeidet med en strategi for bedriftskulturen avhengig av
hvordan Kantega oppfatter strukturen sin, og det er derfor viktig at de ser på
alle apsketer ved strukturen sin for å kunne tilpasse strategien best mulig."
(Kantega-analyse, gruppe 3, s38). Det gruppe 3 poengterer er veldig riktig. En
naturlig tilpassning til dette vil være å benytte scrum-team-flatheten, med en
sterkere tilnærming til adhockratiet for å redusere avstanden mellom utviklere
og ledere.

\section{Rekruttering av ledere}\label{rekruttering}
mangfoldet er ikke i fokus, men de vet at mangfoldet kan bli bedre. mangfoldet
blir likevel nedprioritert til fordel for kompetansen til de man ansetter. 

Selv om ikke mangfoldet er i fokus når Kantega ansetter, så vet de at
mangfoldet har forbedringspotensiale. Kompetanse og personlighet er viktig i
ansettelsesprosessen. 

"En løsning kunne vært å ansette ledere fra eksterne kilder, dyktige ledere som
er der for å lede ikke for å utvikle. På den måten får man ikke problemet med
at noen utviklere ikke er gode nok og blir ledere."(Kantega-analyse, gruppe 3,
s26).

Rekruttering av leder er vanskelig, men man kan tilpasse det. Det kan se ut som
Kantega vil ha ledere som også har erfaring med utvikling. 
Ekstern rekruttering av profesjonelle leder gir, ut fra denne antagelsen,
ingen fordeler. Det som derrimot er gunstig er å finne personer med interesse, og
helst erfaring, innenfor ledelse og utvikling. Om man finner dem internt, eller
henter inn nye etenfra har liten betydning. Man bør heller se på de egenskapene
man mangler i Kantega. Leder som bare er til stede for å lede, ikke for å
utvikle, tilfører ikke kjernevirksomheten, utvikling, i Kantega noe ekstra. 

Gruppe 3 sier at rekruttering av ekstern kontra interne ledere essensielt
handler om effektivitetsøkning, og potensielt negative kulturreaksjoner
(Kantega-analys, gruppe 3, s32). Vidre sier de også at "rekruttering for å øke 
mangfoldet gir økt kreativitet." (Kantega-analyse, gruppe 3, s35). 
Den, potensielt, økte kreativiteten og effektivitetsøkningen 
kan gi flere konflikter, og undergrave de rekrutteringsverdiene man har i dag.
Man børe være forsiktig med å endre ting som allerede fungerer.

En langsiktig løsning kan være å ansette utviklere med interesse for, og
erfaring med, ledelse. Da rekrutere man mindre utviklerrettet, men mer ledelsesrettet. 
Så man får folk som er interessert i prosjektledelse, ledelse og strateg,
samtidig som de har erfaring med utvikling. 

\section{Ledelse og lederrollen}\label{ledelse}
Kan lederrollen være et ansvarsområde på lik linje med
utviklere? Og hvordan påvirker makt dette? 

% bruken av makt.
Gruppe 3 beskriver makt i Kantega hovedsaklig av 'ekspertmakt' typen, hvor
personen som kan mest om et tema har mest makt i den sammenhengen. Vidre i
maktkonteksten trekkes utviklernes oppfatning av lederne fram som 'ledere med
dårlig utviklerevne', altså en negativ kontekst. Her neglisjeres det faktum at
alle ansatt i Kantega er kunnskapsarbeidere med ekspertmakt på sitt spesifikke
område. Tilpasset ledelseskonteksten kan vi se at ledere også kan ha samme
ekspertmakt som utviklerne, bare på sine respektive områder. 

Antagelsen om at alle lederne er avdankede utviklere som ikke var gode nok på
det tekniske er svak. Man må ta høyde for personlige interesser, egenskaper og
ferdigheter. Noen er flinkere, og mer interessert, til å drive med ledelse.

%"Men at lederne ikke nødvendigvis skal være sterke tradisjonelle ledere betyr
%ikke at lederne bare skal være avdankete utviklere, dette er noe Kantega bør
%vurdere..."
%(Kantega-analyse, gruppe 3, s32). 

Alle har HR-ansvar, også kjent som personalansvar, eller kjent som at alle har
ansvar for trivselen i bedriften og at man når dagens visjon om å være verdens
beste arbeidsplass.

"Lederrollen er noe uklar." (Kantega-analyse, gruppe 3, s36). Det har gruppe 3
rett i. Kantega burde definere ledere klarere. En måte å gjøre det på er som
tidligere nevnt ansvarsområder. Om man ser bortifra hvem som er over hvem i et
hierarki kan man fokusere på arbeidsoppgavene, med tillhørende ansvarsområde.
Et slikt område vil være koordinering mellom prosjekter, og forskjellige
entiteter i Kantega. 

Et eksempel på et ansvarsområde er HR-ansvaret. HR-ansvar er en naturlig del av
de å lede. Gode ledere vet hvordan man drive med HR. Man
trenger ikke eksplisitt å definere en person som har personalledelse som arbeidsoppgave.
Når det ikke er definert vil noen ta initiativet til ansvar, og forbedre
situasjonen. 

"Vekst uten klare roller kan føre til en hodeløs kjempe", men det kan også føre
til et velfungerende nettverk av entiteter i Kantega. 

\section{Konklusjon}\label{konklusjon}

Diskusjonen så langt har dreid seg om hvordan Kantega kan utvikle strukturen i
en utflatende rettning, at ekstern rekruttering ikke nødvendigvis er bedre enn
intern, og at lederrollen i sammenheng med struktur kan utvikles til at alle i
Kantega føler mer ansvar og handlingsfrihet. 

Som Gruppe 3 påpeker i sin empiri del, "Kantega opplever at det blir satt av for liten tid til forbedringsforslag fra
de ansatte. Wulff nevner i bedritsfbesøket at: 'de mest grunleggende
ledelsesutfordringene ligger i at de ansatte ønsker å forbedre mye, og at det
er for lite tid for ledelsen å utføre det'(K. Wulff,
05.09.14)"(Kantega-analyse, gruppe3, s4), ser vi at
dagens struktur ikke fungerer optimalt. Det kan se ut som man forventer at
lederene gjør mer enn de har tid til. Hver enkelt kan kanskje også føle at man
ikke har mandat til å forbedre noe utover sin stilling. Har kan man vurdere å
bryte ned grensene enda mer enn man har gjort i dag, og si at alle, ikke bare
ledelse, har ansvar for å forbedre bedriften. Det er ingenting som tilsier at
hver og en ikke kan bidra utover sine hovedoppgaver, noe man burde oppfordre
til.

Rekruttering av ledere fungerer som det er i dag. Kantega har god erfaring med
å rekrutere leder intern, og så vidreutdanne dem. Dette burde man holde på. Man
kan også vurdere å rekrutere langsiktig, hvor kandidatene med interesse for
ledelse kommer høyere på lista.  

Ledelsen i Kantega kan forbedre ved å fokusere på og bedre
spessifisering av ansvarsområder og reduksjon av barrierer. Da vil utviklere og
leder komme nærmere hverandre hierarkisk.  


\section{Referanser}
\begin{itemize}
	\item Kantega-analyse, Gruppe 3, Høst 2014, TIØ4161, NTNU, Trondheim.
	\item Bolman, L. \& Deal, T. 'Nytt perspektiv på organisasjon og ledelse',
4.utg. Oslo, Gyldendal Akademiske Forlag, 2009. 
	\item Wilton, N. An introduction to human resource management, 2nd
edition, London, SAGE.
	\item Sommerville, Software Engineering 9th edition, Pearson, Boston, 2011.
\end{itemize}

\end{document}
This is never printed
