% latex article template

% cheat sheet(eng): http://www.pvv.ntnu.no/~walle/latex/dokumentasjon/latexsheet.pdf
% cheat sheet2(eng): http://www.pvv.ntnu.no/~walle/latex/dokumentasjon/LaTeX-cheat-sheet.pdf
% reference manual(eng): http://ctan.uib.no/info/latex2e-help-texinfo/latex2e.html

% The document class defines the type of document. Presentation, article, letter, etc. 
\documentclass[12pt, a4paper]{article}

% packages to be used. needed to use images and such things. 
\usepackage[pdfborder=0 0 0]{hyperref}
\usepackage[utf8]{inputenc}
\usepackage[english]{babel}
\usepackage{graphicx}
\PassOptionsToPackage{hyphens}{url}

% hides the section numbering. 
\setcounter{secnumdepth}{-1}

% Graphics/image lications and extensions. 
\DeclareGraphicsExtensions{.pdf, .png, .jpg, .jpeg}
\graphicspath{{./images/}}

% Title or header for the document. 
\title{
	Budsjett og Regnskap, Øving 3. Ledelse i Praksis.
}
% Author
\author{
	Magnus L Kirø \\
	IT-sjef, Studentmediene i Trondheim % in what capacity are you presenting this document? as yourself, the sales manager, ceo, etc? 
}
\date{\today}

\begin{document}
\maketitle
\pagenumbering{arabic}

I denne øvingen tar jeg utgangspunkt i IT-budsjettet i Studentmediene i 2013,
og tillhørende skyggeregnskap. 

\section{Nøkkel tall}
Nøkkeltallene fra bedsjettet er Støtte fra eiere, iBok inntekt, iBok fondet, forpleining,
annen driftskostnad, produkttjener, driftsresultat. De fleste postene er slevforklarende.
Annen driftskostnad er alle summen av alle driftsutgifter. Produkttjener er
innkjøp av ny server. 

De totale utgiftene er oppsummert i posten: annen driftskostnad. Annen
driftskostnad summeres til rett i overkant av 165 tusen. De sammenlagte
inntektene summeres til 187 tusen. 

Overensstemmelsen av regnskap og bedsjett er for det meste godt. Med unntak av
iBok inntekter og fond. IBok tjente vesentlig mer penger enn antatt. Dette
fører til at IT ikke trenger posten 'støtte fra eiere'. IT-avdelingen derfro
gått i overskudd.  

Grunnet omstrukturereinger er budsjettet og tillhørende skyggeregnskap, ikke
representativt som mal for 2014. 
Ibok inntaktene har gått over all forventning. 

De to store avvikene er utgifter til iBok fondet go innektene til iBok. 

I den mellomfasen studentmediene er i har vi ikke tidligere hatt egent bedsjett
for IT. Og vi vil ikke ha det igjen neste år. 

\section{Mulige forbedringer}
Det er enkelt å finne tallene i budsjettet. Dette har meg å gjøre at jeg har
brukt tid på å sette meg inn i budsjette og finne ut hva postene er tenkt brukt
til. 

I studentmediene har vi gjennomført en budsjettsevaluering og satt opp nye
budsjetter. I denne prosessen ble IT-budsjettet avviklet og inlemmet i 'IT- og
Driftsbudsjettet'. Dette er gunstig fordi IT-avdelingen bare sitter igjen med
forpleiningsposten etter omorganisering. 

\section{Målsetninger}
Studentmediene har som målsetning å gå i null. Når det er sagt så budsjeterer
vi pessimistisk, slik at vi ikke går i minus. Per i dag er dette viktig da våre
største inntekter(annonseinntekter) er veldig usikkre.

Utover de ovennevnte målene har vi satt noen andre mål for inntekter. Det er
hovedsaklig iBok inntekter budsjetert til 300 tusen og forpleininger budsjetert
til 25 tusen. 

Målsetningene gjenspeiles i inntekt- og utgiftspostene. 

De økonomiske målsetningene kommer fram av behov. Vi ser på de utgiftene vi har
og de inntektene vi har. Og deretter fordeles midlene utover postene. Kritiske
poster først, mindre kritiske etterpå.  

\section{Insentiver og bruk}
Økonomien har en splittet viktighet for gjengen. På den ene siden avhenger
store deler av produskjonen på inntektene. Mens den andre delen ikke avhenger
så mye av inntektene. De rent driftsmessige utgiftene er relativt små og kan
dermed betjenes med moderate inntekter. De produksjonsrelaterte utgiftene er
veldig avhengig av inntektene, slik at man ved lav inntekt vil ha lavere opplag
av Under Dusken, og man må kanskje kutte i andre medieavgifter. 

Budsjettet er mest et styringspinn på de investeringer og det vedlikeholdet vi
kan gjennomføre. Dette mest for drift. Budsjettet bestemmer også kvalitet og
opplag. Ved tvister og prioriteringsproblemer kan man bruke poster på
budsjettet som en avgjørende faktor i bestemmelsesprosessen.   
Budsjettet tilrettelegger også sponsing av sosiale arrangementer. Noe som igjen
insentiverer sosiale sammenkomster.  

I IT, hvor den største posten på budsjettet er forpleining, ligger bruken av
budsjettet hovedsaklig på sosialisering og arbeidstilrettelegging. 
Personlig bruker jeg budsjettet som et middel for å skape indentitet og
tilhørighet innad i IT.

\end{document}
This is never printed
