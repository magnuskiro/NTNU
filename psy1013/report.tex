% cheat sheet(eng): http://www.pvv.ntnu.no/~walle/latex/dokumentasjon/latexsheet.pdf
% cheat sheet2(eng): http://www.pvv.ntnu.no/~walle/latex/dokumentasjon/LaTeX-cheat-sheet.pdf
% reference manual(eng): http://ctan.uib.no/info/latex2e-help-texinfo/latex2e.html

% The document class defines the type of document. Presentation, article, letter, etc. 
\documentclass[12pt, a4paper]{article}

% packages to be used. needed to use images and such things. 
\usepackage[pdfborder=0 0 0]{hyperref}
\usepackage[utf8]{inputenc}
\usepackage[english]{babel}
\usepackage{graphicx}
\PassOptionsToPackage{hyphens}{url}

% hides the section numbering. 
\setcounter{secnumdepth}{-1}

% Graphics/image lications and extensions. 
\DeclareGraphicsExtensions{.pdf, .png, .jpg, .jpeg}
\graphicspath{{./images/}}

% Title or header for the document. 
\title{
	PSY1013, Neural pathways and Pain
}
% Author
\author{
	Magnus L. Kirø \\
}
\date{\today}

\begin{document}
\maketitle
\pagenumbering{arabic}

\section{Pain}
Pain is a cultural thing. In the sense that culture affects the perception of
pain greatly. In some cultures hanging from metal hooks, through the skin on your
back, sends the practitioner into an exalted state. In more sexual sub cultures
pain is used to relief and care. The pain is more of a tool to connect to one
another. 

The main objective of pain is to keep humans alive, and safe. Touching something warm we
have to be able to remove ourselves from heat that is to hot for our body to
handle. Or we have to stop working out when we are injured. Normal pain
indicators for humans to relax more are, among others, headaches, backaches and stomachaches. 

Pain is registered through the somatosensory system. This system registers
touch, temperature and pain. Neurons are triggered in the system when stimulus
is present. The neuron triggered signal is passed to the brain. More
specifically the part of the brain that deals with this kind of input. Each
body part has it's own section in the brain.  

Receptors of pain are called nociceptors. Nociceptors sense change in heat or
pressure. If you touch hot objects the receptors fire. Or if you cut yourself
on knives or other sharp objects. Nociceptors respond to mechanical, thermal,
and chemical stimulation. The chemical stimulation is typically after a work
out session, where the lactic acid burns in you muscles, and the brain
registers soreness.   

\section{Pathways to the brain}
Pain is transferred to the brain in two ways, the fast and the slow one. Both
are through fibres. The fast fibres are called myelinated Aδ fibres. Aδ
transfer short sharp pains, where response time is critical. C fibres,
unmyelinated, transfer slower pain, such as aches and other dull pain. 

The pain signals are sent through the dorsal root in the spinal cord. In the
spinal cord the fibres synapse in the substantia gelatinosa. These synapses
release glutamate and are capable of releasing a neurotransmitter substance
called 'substance P' in the dorsal horn. Substance P stimulates change in the substantia
gelatinosa, which might explain how people adapt to pain.

From the substantia gelatinosa the fibres join the spinothalamic pathway that
runs, along the spinal cord, to the brain. The fibres terminate in the thalamus
in the brain. Pain and temperature information travel along the trigeminal
nerve and synapse in the spinal trigeminal nucleus. 

The thalamus is also connected to the anterior cingulate cortex(ACC). The
somatosensory cortex is also connected to the thalamus, but to a lesser degree.
ACC helps with the emotional part of pain. If we expect low levels of pain ACC
is less active, and if we expect much pain ACC is more active. 

Reward circuits, such as nucleus accumbens, have been observed active in relation
to pain. Nucleus accumbens is essential to addictive behaviours. Aversive, and
positive stimuli are analysed in in reward circuits and have an impact on
survival. 

\section{Pain management}
A negative side of pain is it's persistence. Some people feel so much pain that
they cannot function in society. Endogenous opioid receptors along with
individual differences and culture affects how people experience pain. A person
with high levels of opioid receptors experience less pain. 

We try to stop pain from reaching the brain and affecting us. If we hit our
elbow, we might rub it to lessen the pain. What we in essence do is that we
block the nociceptors from sending signals. This phenomena is often referred to
as the 'gate theory of pain'. In essence it's the activation of touch fibres
that reduces the amount of information the reaches the brain. 

Periaqueductal gray(PAG) is a central place for pain reception. PAG has a high
density of opiate receptors. PAG is like a load balancer. It has possibility to
change incoming data, and selectively relaying information to the brain.  

Chronic pain is a troubling type of pain. It doesn't go away, and many types of
medicine have little effect on it. Chronic pain is said to be more of a memory
problem than a sensory one. It is believed that the problem is forgetting the
pain. 

Further we have pain during stress. In very stressful situations the feeling of
pain is reduced. This is done by the increased levels of opioids in the brain,
mainly endorphins. 

Attitudes towards pain also plays a big role in how we handle it. As an example
athletes and nonathletes experience pain very differently, although their pain
thresholds can be the same. The sense of control reduced how much pain a person
feels. Or rather how much pain a person can live with. Self medication for pain
results in less medication.  

\section{References}
Lecture slides. 

Discovering Biological Psychology, Second edition, published by Wadsworth,
ISBN-13: 978-0-547-17795-3, pages: 202-223. 


\end{document}
