% latex article template

% cheat sheet(eng): http://www.pvv.ntnu.no/~walle/latex/dokumentasjon/latexsheet.pdf
% cheat sheet2(eng): http://www.pvv.ntnu.no/~walle/latex/dokumentasjon/LaTeX-cheat-sheet.pdf
% reference manual(eng): http://ctan.uib.no/info/latex2e-help-texinfo/latex2e.html

% The document class defines the type of document. Presentation, article, letter, etc. 
\documentclass[12pt, a4paper]{article}

% packages to be used. needed to use images and such things. 
\usepackage[pdfborder=0 0 0]{hyperref}
\usepackage[utf8]{inputenc}
\usepackage[english]{babel}
\usepackage{graphicx}
\PassOptionsToPackage{hyphens}{url}

% hides the section numbering. 
\setcounter{secnumdepth}{-1}

% Graphics/image lications and extensions. 
\DeclareGraphicsExtensions{.pdf, .png, .jpg, .jpeg}
\graphicspath{{./images/}}

% Title or header for the document. 
\title{TDT4300 Datavarehus og datagruvedrift
\\
Assignment 3: Classification}
% Author
\author{
        Magnus L Kirø \\
}
\date{\today}

\begin{document}
\maketitle
\pagenumbering{arabic}

\section{1 Decision Trees}
\begin{itemize}
	\item 1: tot gini= 0.930801388889
	\item 2: id : gini= 0.95
	\item 3: age : gini= 0.665
	\item 4: student : gini= 0.5
	\item 5: creditworthiness : gini= 0.5
	\item 6: The best gini value is the total for the whole set. Besides that
age is the best gini value. The id(UserID) is a unique value and would in many
cases be useless when searchinf for information. 
\end{itemize}

The two last gini indexes are "income : gini= 0.645" and "pc : gini= 0.48"

\section{2.2 datasets}
%How many variables do they contain?
%How many instances?
%What kind of data do they describe?
%Which one one do you assume is the most difficult to classify?

\begin{itemize}
	\item Iris dataset \\
	\subitem The dataset contains three (3) variables. 
	\subitem 150 instances are available. 
	\subitem The variables describe, petal length, petal width and class. 
	\subitem I assume the petal width is the most difficult to classify. 	

	\item Diabetes dataset \\
	\subitem Has nine attributes. 
	\subitem 768 instances are present in the dataset. 
	\subitem The data describes the condition and stats of diabetes patients. 
	\subitem I would presume that the \textit{insu} variable is the hardes to
classify. 

	\item Spambase dataset \\
	\subitem 58 variables can be found in the dataset.
	\subitem 4601 instances ar present in the dataset. 
	\subitem The content of the dataset describes content of spam email. 
	\subitem Class as a variable here would be the most difficult one to
classiy. 

\end{itemize}


\section{2.3 Classification}
%List the most important parameters.
%Describe how changing them changes the overall result. 
%Overal result/accuracy. 


List the algorithems, with corresponding findings. 
\begin{itemize}
	\item J48
	\subitem Iris; The petal width is the most important variable.
The accuracy is 96.08\%
	\subitem Diabetes; \textit{Plas} is the most important variable here.
The accuracy is 76.25\%
	\subitem spambase; \textit{word\_freq\_remove} is here the most important
variable. Accuracy=92.20\%

	\item k-NN
	\subitem Iris; Using the neares neighbour the petal width variable is the
most significant one. accuracy=96.08\%
	\subitem Diabetes: the plas variable would still be the most significant
one. accuracy=72.80\%
	\subitem spambase; The most deciding variable is \textit{word\_freq\_remove}
and the accuracy is 89.0\%
	

	\item Support vector machines
	\subitem Iris; petal width has the most weight here. accuracy=96.08\%
	\subitem Diabetes; Plas has the most significant value here. accuracy=79.31\%
	\subitem spambase; \textit{word\_freq\_remove} is still the most
significant variable accuracy=90.54\%
\end{itemize}


\subsection{2.4 Evaluation}
Cross-validation is the process of splitting the dataset into subsets,
execute calculations, then comparing all the calculation of the subsets to
create a better total estimate of the calculation. 

With the k-NN algorithm the cross-validation gives the biggest difference in
accuracy. 

Cross-validation will be a better choice with bigger datasets and more complex
datasets. Splitting the calculations into sub calculations can save time and
complexity, thus reducing resource and time usage. And the over all result
becomes a bit more accurat.

In the case where the data is sorted the percentage split won't help much. Then
the subsets would be to different to give good results. Having four fruits, two
apples and two oranges, there would be little point in dividing them into two
sets where the same fruits are in the same sets. Splitting like this,
calculating and then combine would only result in bigger resource usage and
time delay. 

\section{2.5 Best Classifiers}

J48 showed the best performance with all datasets. 
The REPTree algorithm showed better results then any of the others. 

\end{document}
