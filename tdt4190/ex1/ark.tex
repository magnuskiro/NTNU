\documentclass[12pt]{article}
\usepackage[utf8]{inputenc}
\usepackage[english, norsk]{babel}

\begin{document}
    Distsys ex1 - RMI \\
    Jan Alexander Bremnes & Magnus Kirø \\ 

\section{Server}    
        How the server side sets up the RMI part of the code. \\

        Creates the user interface class that manages the game.\\
	    \underline{code:} TicTacToeGui serverGui = new TicTacToeGui("Server", 'X', null);
			
        Creates the communication instance. The class that is used for communication \\
		\underline{code:} TicTacToeImpl impl = new TicTacToeImpl(serverGui);
		
		Make the communication instance visible to clients that connect.\\
        \underline{code:} Naming.rebind("rmi://"+adresse+"/TicTacToe",impl);
	
\section{Client}
	    What the client does to make the RMI connection work. \\
	
	    This is the address that will be bound to the RMI connection. the address is input from command line.\\
    	\underline{code:} String url = "rmi://"+adresse+"/TicTacToe"; 
    	
        Binds the servers communication class to the clients server variable. This enables us to call methods in the servers calass. \\
		\underline{code:} TicTacToe server = (TicTacToe)Naming.lookup(url);

		Setup of the lients communication class. This is the calss that the server can connect to execute methods on the client. \\
        \underline{code:} TicTacToeImpl client = new TicTacToeImpl(clientGui);

		Sets the clients communication instance at the server. This gives the server the reference to the clients communications class, so the server can perform actions on the client side. \\
        \underline{code:} server.setClient(client);

\end{document}
